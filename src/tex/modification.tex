\section{La modification non-destructrice}
	\begin{frame}{La modification non-destructrice}
	\framesubtitle{A quoi ça sert?}
	\begin{itemize}
		\item On peut revenir en arrière après avoir enregistré et rouvert son fichier
		\item Plus de possibilité de réparer ses erreurs
		\item Plus de possibilité de ne pas avoir à recommencer depuis le début...
		\item On évite les pertes d'information liées à la compression (jpeg par exemple)
	\end{itemize}
	\end{frame}

	\begin{frame}{La modification non-destructrice}
	\framesubtitle{Comment ça fonctionne?}
	Le logiciel ne va plus directement altérer les pixels, mais enregistrer les modifications dans des fichiers annexes ou dans le document même (le format .xcf, pour gimp).
	\begin{itemize}
	\item On va utiliser au maximum les calques, quitte à en abuser
	\item Lâcher la gomme au profit des masques
	\item Appliquer localement les modifications avec les deux premiers points
	\end{itemize}
	\end{frame}
