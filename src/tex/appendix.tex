%%%%%%%% APPENDIX %%%%%%%%
\appendix


\begin{frame}{Quelques liens utiles}
	\begin{itemize}
		\item Documentation en français : \textcolor{webblue}{\url{https://docs.gimp.org/fr/}}
		\item Tutoriels en anglais : \textcolor{webblue}{\url{https://www.gimp.org/tutorials/}}
		\item Compilation de tutoriels en anglais : \textcolor{webblue}{\url{http://www.gimpology.com/}}
		\item Raccourcis : \textcolor{webblue}{\url{http://www.gimpusers.com/gimp/hotkeys}}
	\end{itemize}
\end{frame}


%\begin{comment}
\begin{frame}
	\textbf{Récapitulatif des raccourcis}
	\begin{itemize}
	%todo A CLASSER (ex: view, selection, tools, ...)
	\item \keys{Ctrl + B} : Boîte à outils
	\item \keys{Ctrl + L} : Fenêtre de Calques
	\item Raccourcis classiques
	\begin{itemize}
	 \item \keys{Ctrl + A} : Tout sélectionner
	 \item \keys{MAJ + Ctrl + A} : Tout déselectionner
	 \item \keys{Ctrl + C} : Copier
	 \item \keys{Ctrl + V} : Coller
	 \item \keys{Ctrl + I} : Inverser la sélection
	\end{itemize}
	\item \keys{{+}} : Zoom (+)
	\item \keys{-} : Zoom(-)
	%\item \keys{1},\keys{2},\keys{3},\keys{4},\keys{5}, \keys{MAJ + 2}, \keys{MAJ + 3}, \keys{MAJ + 4}, \keys{MAJ + 5}
	\item \keys{R} : Sélection rectangulaire
	\item \keys{E} : Sélection élliptique
	\item \keys{F} : Sélection libre
	\item \keys{Ctrl} (appuyé) = Sélection de couleur
	\end{itemize}

	\begin{center}
	\fbox{\url{http://www.gimpusers.com/gimp/hotkeys}}
	\end{center}
\end{frame}
%\end{comment}

%\begin{frame}
%	G'MIC
%\end{frame}

\begin{comment}
\section{Comment}
\begin{frame}
	Le libre rapidement

	Permet de faire plein de choses cool (et rapides)

	Audacity,...
\end{frame}

\begin{frame}

	\begin{enumerate}
	\item Ouvrir une image : File/open ==> choisir l'image à ouvrir

	Ou ... plus simple : faire glisser l'image sur l'environnement de travail

	\item Créer un nouveau projet ==> format .xcf

	\item Pour obtenier une image en format "commun" ==> File/Export as ==> choisir le nom de fichier et l'extension (.png, .jpeg, .pdf, .text, .gif, .ps, .psd(compatibilité photoshop),...)
	\end{enumerate}
\end{frame}

\begin{frame}
	Description rapide des fenêtres
	\begin{enumerate}
	\item Menu
	\item Fenêtre d'outils
	\item Fenêtre de calques

	\end{enumerate}
	Environement ouvert (échelles, zone de travail)
\end{frame}

\begin{frame}
	Outils : description générale

	Ctrl + B pour l'ouvrir (ou windows/Toolboxes)

	\begin{enumerate}
	\item Double clique pour les paramètres
	\end{enumerate}
\end{frame}

\begin{frame}
	Outils : description par outil

	+ Raccourcis de l'outil

	+ Paramètres les plus couramment utilisés
\end{frame}

\begin{frame}
	Crayon

	Majuscule maintenue + Clique ==> tracer une ligne droite
\end{frame}

\begin{frame}
	Layer manager

	Ctrl + L (ou windows/Layers) pour l'ouvrir

	$\Rightarrow$ pour masquer, superposer, mettre à l'avant plan,...

	\begin{itemize}
		\item Calque (tel quel)
		\item Texte
		%\item Filtre (déconseillé)
	\end{itemize}

	Attention : modifications QUE sur les layers sélectionnés

	+ Masque (= ...)
	\todo[inline]{exemple photo (layes cachés,...)}
\end{frame}

\begin{frame}
	Selection editor (All, shrink, grow, invert,...)

	$\Rightarrow$ Clique droit/	Select/All,Invert,None,...
\end{frame}

\begin{frame}
	Canaux (alpha et différentes couleurs)

	Canal alpha = ??

	Opérations sur les couleurs (mentionner rapidement : niveaux, balances, seuils, saturation,...)
\end{frame}

\end{comment}
