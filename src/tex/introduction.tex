\section{Introduction}
\begin{frame}{Qu'est-ce que GIMP ?}
	\begin{overprint}
	\begin{enumerate}
		\only<1>{
			\item[]
				\begin{itemize}
				    \item GIMP = GNU Image Manipulation Program
				    \item Logiciel de traitement d'images matricielles
				    \item Alternative gratuite et libre à Adobe Photoshop
				\end{itemize}
				\begin{figure}
				    \centering
				    \includegraphics[width=0.5\textwidth]{Images/gimp-logo}
				    \caption{Wilber, la mascotte officielle de GIMP}
				\end{figure}
		}

		\only<2>{
			\item[]
				Liste non exhaustive des fonctionnalités de GIMP
				\begin{itemize}
				    \item Gestion des calques;
				    \item Plusieurs outils de dessin;
				    \item Plusieurs outils de sélection;
				    \item Outils de transformation;
				    \item Une sélection de filtres;
				    \item Gestion de nombreux formats (JPEG, GIF, PNG, PSD, TIFF);
				    \item ...
				\end{itemize}
		}

	\end{enumerate}
	\end{overprint}
\end{frame}

\begin{frame}{Installation}
	\begin{center}
	GIMP est disponible pour Windows, Linux et macOS
	\vspace{1cm}\Large
	\fbox{\url{https://www.gimp.org/downloads/}}
	\end{center}
\end{frame}
